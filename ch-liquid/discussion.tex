\section{Discussion and Summary}
\label{sec:liquid:discussion}

We have tuned the $E_F$ of the surface Dirac fermions over a considerable range, by applying the novel ionic liquid gating technique to the insulating Bi$_2$Te$_2$Se crystals with $\rho >$ 4 $\Omega$cm at 5 K.
In contrast to previous gating experiments, we readily resolve prominent surface SdH oscillations 
at each gating voltage. From the SdH periods, we find that the surface Fermi energy $E_F$
(Sample 1) decreases  
from 180 mV to 75 mV above the Dirac Point as $V_G$ is changed from 0 to -2.8 V. 
In a 14 T magnetic field, the lower limit corresponds to
the middle of the broadened $N$ = 1 Landau Level. Attaining such low Landau levels enables the $-\frac12$
intercept (predicted for Dirac fermions) to be determined with high accuracy.
Furthermore, the intercepts are closely similar for a broad range of $V_G$ in both Samples 1 and 2, adding more evidence for a $\pi$ Berry phase of the surface Dirac electrons on Bi$_2$Te$_2$Se.

Through analyzing the SdH oscillations, we also obtain the surface mobility $\mu_s$ and density $n_s$ for the surface states. And our calculation shows that the Hall conductivity from the surface Dirac fermions exposed to the anions accounts
for up to 83$\%$ of the total observed weak-$B$ Hall conductivity at 5 K. Such a large proportion is mainly caused by the large difference between the surface and the bulk mobilities. Combining these parameters with a semi-classical two-band model, we obtain an accurate determination
of the bulk carrier mobility and density at each $V_G$ (Fig. \ref{figHall}). We find that the surface carrier density $n_s$ decreases steeply while the surface mobility
$\mu_s$ increases to a maximum value of 2,400 cm$^2$/(Vs) under gating. The bulk carriers are depleted to a depth of 10 $\mu$m from the 
surface, with $\mu_b$ remaining at the low value of 20 cm$^2$/(Vs).



The large enhancement of $\mu_s$
by liquid gating (Fig. \ref{figHall}b) is perhaps the most intriguing 
feature in our ionic liquid gating experiments. To our knowledge, this is the first realization of 
enhancement of surface SdH amplitudes by an \emph{in situ} technique, and may provide a way to improve the surface conductivity. A clue for the explanation arises from an STM experiment~\cite{Beidenkopf2011}, which reveals that
the Dirac Point closely follows spatial fluctuations of the local
potential on length scales of 30-50 nm. Such a large fluctuation in the surface band can lead to strong scattering
of surface electrons and then reduces the surface mobility. We speculate that, under liquid gating, the anions accumulate
at local maxima in the potential, thereby leveling out the strongest spatial fluctuations and improving the electron lifetime.
The results are encouraging and indicate that alternative routes that even out local
potential fluctuations can further improve $\mu_s$. This could be important for future applications and devices.

We also provide analysis and experimental data in order to uncover whether the ionic liquid gating actually alters the carrier concentration by chemical
reactions or simply through bending the band. 
By carefully selecting the experimental conditions (e.g. the gating temperature),
monitoring charge accumulated $Q$, and checking for reversibility, we establish that band-bending is the dominant effect
in our experiments. Lastly, the five quantities measured at each gate voltage setting ($E_F$, $n_s$, $n_b$, $\rho$ and $Q$)
provide a quantitative picture of the gating process and the band bending effect. The depletion
capacitance measured implies that, within the depletion region, the electronic polarizability is strongly enhanced.
