\section{Discussions and Summary}
\label{sec:bts:summary}


Many non-trivial properties of the Dirac-like topological surface states have been investigated by the surface sensitive ARPES and STM experiments. Although the large amount of bulk carriers may overwhelm the fragile surface states on many TIs, here we found a TI crystal with a large bulk resistivity with the appropriate charge compensation method. The suppressed bulk conductance has enabled us to measure the surface states of Bi$_2$Te$_2$Se directly in a transport experiment. We showed that these surface states on Bi$_2$Te$_2$Se may also display interesting physics for the transport experiments at high magnetic fields. We can use the powerful SdH oscillations to identify the surface conductance and explore the surface states' properties such as $k_F\ell$ and the mobility. With a magnetic field up to 45T, our Bi$_2$Te$_2$Se samples could access $n$=1 or lower Landau levels, providing an accurate determination of the intercept $\gamma$. The intercept we obtained by linearly fitting the Landau index plot $n$ versus $1/B_n$ is close to $-\frac12$. It corresponds to a non-trivial $\pi$ Berry phase expected for the Dirac dispersion of the surface states on TI.  

We also hope to emphasize that the appropriate definition of $B_n$ should be applied to exclude any mistakes in identifying $\gamma$. As explained in the above sections, $B_n$ should be identified with minima in $G_{xx}$ or maxima in $R_{xx}$ when the bulk conductance dominate. In fact it is very tempting to isolate oscillatory $\Delta R_{xx}$ by subtracting a smooth background and then assign integer indices to the minima of the $\Delta R_{xx}$. Nevertheless, this way will result in an opposite and wrong definition of the index. Then it will yield a fake $\frac12$-shift in the intercept even for Schr\"{o}dinger electrons. Furthermore it needs a low index to significantly reduce the uncertainties in fixing $\gamma$. An index plot with many but large indices may still have a large error bar in determining $\gamma$ as the curvature in the plot could easily change the intercept obtained by extrapolation. 

The linearity of the index plot in Figs. \ref{figindex} and \ref{figG1} show that the Lande
$g$-factor can not be large ($g\sim$2). The $n$ = 0 LL is unshifted even at 45 T, consistent with 
STM experiments taken at 11 T~\cite{Hanaguri,Xue10}.

Finally, we hope to add some comments on the results in the large-$B$ limit.
In Fig. \ref{figG}a, the last maximum in $\Delta G_{xx}$ (at $B\simeq$ 40 T) corresponds to
$n$ = $\frac12$ (see arrow in the index plot in Fig. \ref{figindex}c). At this field, the
Fermi energy $E_F$ is aligned with the center of the $n$ = 1 LL, as sketched in the
inset in Fig. \ref{figindex}c. According to our indexing scheme, it means that there are two half-filled
LLs between $E_F$ and the Dirac Point, with $\frac12$ of the $n=1$ Landau level and $\frac12$ 
from the unshifted LL at the Dirac Point. Since there are not many LLs between $E_F$ and the Dirac point, our high-field results provide 
rather firm evidence 
for this $\frac12$-shift in the limit $1/B\to 0$. 
As the inset in Fig. \ref{figindex}c implies, we may be able to reach the interesting
states in the $n$ = 0 LL in Bi$_2$Te$_2$Se in fields higher than 45 T. 
