%This is a \LaTeX{} template and document class for Ph.D. dissertations at Princeton University. It was created in 2010 by Jeffrey Dwoskin, and adapted from a template provided by the math department. Their original version is available at: \url{http://www.math.princeton.edu/graduate/tex/puthesis.html}
%
%This is \textbf{NOT} an official document. Please verify the current Mudd Library dissertation requirements~\cite{mudd2009} and any department-specific requirements before using this template or document class.
%
%
%Your abstract can be any length, but should be a maximum of 350 words for a Dissertation for ProQuest's print indicies (or 150 words for a Master's Thesis); otherwise it will be truncated for those uses~\cite{proquest2006}.
%
%
%Dwoskin Ph.D. Dissertation Template --- version 1.0, 5/19/2010
%
%
%This thesis summarizes the electric transport experimental results on the bulk-insulating topological insulators and three dimensional Dirac semimetals. Unlike the traditional description of crystals, which utilizes the local symmetry, the topological insulator theory utilizes the electron wavefuntion space as a whole to distinguish different insulators. One important prediction of these theories is that topological insulators are supposed to host two-dimensional Dirac states on the surface. These surface states are spin-polarized and can serve as the bricks for novel physics such as axion electrodynamic phenomena and Majorana fermions. 

In recent years, the progress in understanding the Berry phase of Bloch electrons in crystals has triggered tremendous interest in discovering  novel topological properties and exotic phases of solids. In a crystal, the Berry curvature acts as an intrinsic magnetic field when an electron moves in the momentum space. The integration of  the Berry curvature in the Brillouin zone can categorize solids into phases such as topological insulators (TI), Chern insulators, Dirac semimetals (DSM) and Weyl semimetals (WSM). These new phases have unconventional electronic states at the boundaries. For example, a three-dimensional (3D) TI hosts spin polarized electrons on its surface, while chiral fermions live on the magnetic domain boundaries of quantum anomalous Hall phases. These Dirac-like edge states are usually protected by the time-reversal (TR) symmetry and stay immune to the backscattering. Under proper engineering, such edge states can carry a dissipationless current without a magnetic field, leading to a great application potential in low-power devices. In addition, these edge states may serve as building bricks of future topological quantum computers.

Besides TI, the newly discovered Dirac and Weyl semimetals represent another example in which electrons obey equations from high energy physics. Both a DSM and a WSM have a linear energy-momentum dispersion. Besides, a WSM hosts an even number of topologically protected Weyl nodes. The paired Weyl nodes have opposite chiralities, and can be regarded as positive and negative monopoles of the Berry flux. There are also surface Fermi arcs that connect the Weyl nodes. Under the TR, inversion and certain crystal symmetries, as in the cases of Cd$_3$As$_2$ and Na$_3$Bi, the Weyl nodes with different chiralities can coexist at the same point in the Brillouin zone and the crystal becomes a Dirac semimetal (DSM). Such semimetals are predicted to have a charge pumping effect between the different Weyl branches when both an electric field and a magnetic field are present, known as the chiral anomaly.

The above predictions lie at the focus of our experimental study of topological materials. However, the defects in real-life crystals of TI have created a lot of obstacles to the experimental progress. For example, the exotic surface Dirac electrons on topological insulators are usually hidden by the enormous amount of bulk carriers in the transport study. Such bulk carriers are commonly generated by the crystal defects and have hampered the study of the novel transport properties of the Dirac surface states. To reduce the bulk carriers, we used the charge compensation method to synthesize a topological insulator, namely Bi$_2$Te$_2$Se, with a suppressed carrier density. Owing to the bulk density as low as  $10^{16}$ cm$^{-3}$ in Bi$_2$Te$_2$Se, we observed the prominent Shubnikov$-$de Haas oscillations from the surface states on Bi$_2$Te$_2$Se. By analyzing the quantum oscillations, we inferred that the surface mobility is $\sim$ 2800 cm$^2$/(Vs), orders of magnitude larger than the bulk mobility ($\sim$ 50 cm$^2$/(Vs)). By pushing the surface electrons close to the lowest Landau level in a high magnetic field up to 45 Tesla, we demonstrated clear evidence for the $\pi$ Berry phase of the Dirac surface electrons on Bi$_2$Te$_2$Se. Besides the chemical doping method, we also leveraged the ionic liquid gating technique and successfully brought the chemical potential $50 \%$ closer to the Dirac point in Bi$_2$Te$_2$Se. Also, the surface mobility was improved by three times in our ionic liquid gating experiments.

In addition to topological insulators, we also studied Na$_3$Bi, a 3D DSM stabilized by $C_3$ rotational symmetry. Through a careful sample mounting procedure, we avoided any oxidization of the Na$_3$Bi sample, and found that Na$_3$Bi crystals with different levels of the chemical potential $E_F$ have two types of transport behavior. The Na$_3$Bi crystals with a high chemical potential ($E_F$ $\sim$ 400 meV) exhibit a large and linear magnetoresistance (MR) up to 35 T, while the Hall angle has an unusual step-function behavior. Besides, the longitudinal and Hall conductivities share similar power-law dependence in a large magnetic field. These significant deviations from the conventional transport behavior may result from an unusual sensitivity of the transport lifetime to the magnetic field. In the second type of Na$_3$Bi with a chemical potential as low as approximately 30 meV, we observed a novel, negative and highly anisotropic magnetoresistance. Previous theoretical studies have shown that the combined electric $\bf E$ and magnetic $\bf B$ fields can generate axial currents between different Weyl nodes and lead to a negative MR. By rotating both $\bf E$ and $\bf B$, we demonstrated that the negative MR in Na$_3$Bi is consistent with the theoretical prediction of the chiral anomaly effect in the DSM. In addition, we found that the novel negative MR in Na$_3$Bi is very sensitive to the angle between $\bf B$ and $\bf E$. The evidence for the chiral anomaly effect is very inspiring and may encourage a lot of future investigation on the Weyl physics in condensed matter.