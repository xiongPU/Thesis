
\chapter{Conclusion\label{ch:conclusion}}

In this thesis, we have discussed our transport study of two types of Dirac electrons in solids, i.e. the surface electrons in a topological insulator and the 3D electrons in a Dirac semimetal. The electronic states of these two systems can both be described by the Dirac equations, in two dimensions and three dimensions respectively. The nature of these states is deeply connected to the Berry curvature and topology in the momentum space. Such characteristics in the momentum space determine many physical properties of the crystals. Through our high field transport study, we have been able to reveal some of the novel properties induced by the exotic topological structure in these crystals.

By studying a bulk-insulating TI Bi$_2$Te$_2$Se, we have been able to demonstrate prominent quantum oscillations from its Dirac surface states. We have synthesized Bi$_2$Te$_2$Se crystals based on the chemical doping method and have obtained one of the most bulk-insulating topological insulators. Hence, we are able to achieve a high surface proportion in the total conductance. Besides, with a relatively low $E_F$, the sample can enter the $N=1$ Landau level in a high magnetic field up to 45 T. It enables us to measure the Berry phase of the Dirac electrons with a high accuracy. By comparing different samples, our data provide strong evidence for a Berry phase of $\pi$. To even reduce $E_F$, we have leveraged the powerful ionic liquid gating technique. We have successfully increased the sample resistance and decreased the Hall density by a significant amount with a negative gating voltage. Meanwhile, the decreased periods of the surface quantum oscillations suggest that the Fermi surface area has shrunk by approximately 75\%. Notably the amplitudes of the SdH oscillations have grown enormously under gating, implying an improved mobility. These results indicate that ionic liquid gating may yield an effective way to improve the surface contribution in the conductance. To the future researchers working on TI, our study of Bi$_2$Te$_2$Se has provided a promising candidate material for advanced study of TI physics. For example, it is worth looking for the surface quantum Hall effect or quantum anomalous Hall effect in chemically-doped Bi$_2$Te$_2$Se samples under gating. In addition, with samples of a better surface mobility, it is of great value to search for possible fractional quantum Hall effect on the surface of TIs. 

Besides the 2D Dirac electrons, the 3D Dirac fermions in Na$_3$Bi have demonstrated intriguing transport properties in our study as well. We have investigated two types of Na$_3$Bi crystals, one with a high $E_F$ and one with an $E_F$ close to the Dirac point. The first type displays a linear transverse MR up to 35 T. With a comparison to the semiclassical transport theory, we found that it indicates a transport lifetime that is largely mediated by the magnetic field. The second type of Na$_3$Bi has a small bulk carrier density and demonstrates a surprisingly large negative longitudinal magnetoresistance. The conductance converted from the magnetoresistance has a quadratic shape, consistent with the theoretical prediction for the axial current between paired Weyl branches\cite{Son2013}. More importantly, by rotating both $\bf B$ and $\bf E$, we have found that the observed negative MR is very sensitive to any deviation of $\bf B$ from $\bf E$. This is unexpected from conventional transport theory, but is qualitatively consistent with the chiral anomaly effect that pumps charge between Weyl nodes. It also implies plentiful room for both theoretical and experimental study on the angular dependence of chiral anomaly effect when $\bf B$ and $\bf E$ are not parallel. This work provides one of the first evidences for the chiral anomaly effect in a crystal. Therefore, as Na$_3$Bi has proven to be a great platform to study Weyl and Dirac physics, further study of other predicted Weyl physics may be conducted. For example, with magnetically doped Na$_3$Bi, it will be interesting to search for the anomalous Hall and chiral magnetic effects. Besides, it will also be of great value to investigate possible non-local transport phenomena\cite{Parameswaran2014} in the future.

%In this work, we explain how to use the puthesis.cls class file and the accompanying template.   % Conclusion
%\section{Future Work}

Future work should include options in the template for a masters thesis or an undergraduate senior thesis. It should also support running headings in the headers using the `headings' pagestyle.  The print mode and proquest mode included in the template might also be candidates to include in the class itself. 

  % Future work
