%This is a \LaTeX{} template and document class for Ph.D. dissertations at Princeton University. It was created in 2010 by Jeffrey Dwoskin, and adapted from a template provided by the math department. Their original version is available at: \url{http://www.math.princeton.edu/graduate/tex/puthesis.html}
%
%This is \textbf{NOT} an official document. Please verify the current Mudd Library dissertation requirements~\cite{mudd2009} and any department-specific requirements before using this template or document class.
%
%
%Your abstract can be any length, but should be a maximum of 350 words for a Dissertation for ProQuest's print indicies (or 150 words for a Master's Thesis); otherwise it will be truncated for those uses~\cite{proquest2006}.
%
%
%Dwoskin Ph.D. Dissertation Template --- version 1.0, 5/19/2010
%
%
%This thesis summarizes the electric transport experimental results on the bulk-insulating topological insulators and three dimensional Dirac semimetals. Unlike the traditional description of crystals, which utilizes the local symmetry, the topological insulator theory utilizes the electron wavefuntion space as a whole to distinguish different insulators. One important prediction of these theories is that topological insulators are supposed to host two-dimensional Dirac states on the surface. These surface states are spin-polarized and can serve as the bricks for novel physics such as axion electrodynamic phenomena and Majorana fermions. 

The progress in understanding the Berry phase of Bloch electrons in crystals has triggered tremendous interest in discovering novel topological phases of solids. The integration of the Berry curvature in the Brillouin zone can categorize solids into phases such as topological insulators (TI), Dirac semimetals (DSM) and Weyl semimetals (WSM). These new phases have unconventional electronic states at the boundaries, such as the spin polarized electrons on the surface of a three-dimensional TI. Under proper engineering, such edge states can carry a dissipationless current, leading to a great application potential in low-power devices and topological quantum computers.

Besides TI, the newly discovered Dirac and Weyl semimetals represent another example in which electrons have a linear energy-momentum dispersion. The paired Weyl nodes have opposite chiralities, and can be regarded as positive and negative monopoles of the Berry flux. Under the time-reversal, inversion and certain crystal symmetries, as in the cases of Cd$_3$As$_2$ and Na$_3$Bi, the Weyl nodes with different chiralities can coexist at the same point in the Brillouin zone and the crystal becomes a Dirac semimetal. Such semimetals provide platforms for some phenomena in high energy physics, such as the chiral anomaly effect.

The above predictions lie at the heart of our experimental study of topological materials. We synthesized a topological insulator, Bi$_2$Te$_2$Se, with a suppressed bulk carrier density. Analysis of the prominent Shubnikov$-$de Haas oscillations in Bi$_2$Te$_2$Se demonstrates clear evidence for the Dirac surface electrons and their $\pi$ Berry phase. We also leveraged the ionic liquid gating technique to bring the chemical potential $50 \%$ closer to the Dirac point. Additionally, we studied two types of Na$_3$Bi, a DSM. The first type with a high chemical potential exhibits a large and linear magnetoresistance (MR), implying a transport lifetime steeply tuned by the magnetic field. In the second type of Na$_3$Bi with a low chemical potential, we observed a novel, negative and highly anisotropic magnetoresistance. By rotating both the electric and magnetic fields, we demonstrate that the negative MR pattern is consistent with the theoretical prediction for the chiral anomaly effect in a DSM. 