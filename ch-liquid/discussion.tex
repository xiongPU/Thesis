\section{Discussion and Summary}
\label{sec:liquid:discussion}

As we show in the previous sections, we have tuned the $E_F$ of the surface Dirac fermions on our Bi$_2$Te$_2$Se sample over a considerable range, by applying the novel ionic liquid gating technique. In contrast to previous gating experiments, we readily resolve prominent surface SdH oscillations at each gating voltage. More importantly, from the SdH periods, we find that the surface Fermi energy $E_F$
(Sample 1) decreases from 180 meV to 75 meV above the Dirac Point as $V_G$ is changed from 0 to -2.8 V. The powerful gating technique enables the $E_F$ to reach the middle of the broadened $N$ = 1 Landau Level in a 14 T magnetic field. As a result, such low Landau levels makes it possible to determine the $-\frac12$ intercept (predicted for Dirac fermions) with high accuracy.
Furthermore, the intercepts are closely similar for a broad range of $V_G$ in both Sample 1 and Sample 2, adding more credit for a $\pi$ Berry phase of the surface Dirac electrons on Bi$_2$Te$_2$Se.

Through analyzing the SdH oscillations, we also obtain the surface mobility $\mu_s$ and density $n_s$ for the Dirac surface states, and find how they evolve at various gate voltages. Our calculation shows that the Hall conductivity from the surface Dirac fermions exposed to the anions accounts
for up to 83$\%$ of the total observed weak-$B$ Hall conductivity at 5 K. Such a large proportion is mainly caused by the large difference between the surface and the bulk mobilities. Combining these parameters with a semi-classical two-band model, we obtain an accurate determination
of the bulk carrier mobility and density at each $V_G$ (Fig. \ref{figHall}). We find that the surface carrier density $n_s$ decreases steeply while the surface mobility
$\mu_s$ increases to a maximum value of 2,400 cm$^2$/(Vs) under gating. The depletion layer of the bulk carriers is 10 $\mu$m thick from the 
surface, with $\mu_b$ remaining at the low value of 20 cm$^2$/(Vs).



The large enhancement of $\mu_s$
by liquid gating (Fig. \ref{figHall}b) is astonishing to us, and perhaps is the most intriguing 
feature in our ionic liquid gating experiments. Over these years, the TI experiments have been hampered seriously by the low conductivity of the surface states, and the high-mobility surface is desired for future applications.To our knowledge, this is the first realization of 
enhancement of surface SdH amplitudes by an \emph{in situ} technique, and may point to a way to improve the surface conductivity of TIs. A clue for the explanation arises from a recent STM experiment~\cite{Beidenkopf2011}, which reveals that
the surface Dirac Point closely follows spatial fluctuations of the local
potential on length scales of 30-50 nm and 20 meV. Such a large fluctuation in the surface band can lead to strong scattering
of surface electrons and then greatly reduces the surface mobility. We speculate that, under liquid gating, more anions accumulate
at local maxima in the potential, thereby compensating the potential and suppressing the fluctuations while improving the lifetime of surface electrons.
Although more tests for this hypothesis are needed, the results are encouraging and indicate that alternative approaches that even out local
potential fluctuations can further improve $\mu_s$. This could be important for future applications and devices based on TI's surface states.

We also provide analysis and experimental data in order to uncover whether the ionic liquid gating actually alters the carrier concentration by chemical
reactions or simply through bending the band (more details are in the appendix). 
By carefully selecting the experimental conditions (e.g. the gating temperature),
monitoring charge accumulated $Q$, and checking for reversibility, we establish that band-bending is the dominant effect
in our experiments. Lastly, the five quantities measured at each gate voltage setting ($E_F$, $n_s$, $n_b$, $\rho$ and $Q$)
provide a quantitative picture of the gating process and the band bending effect. Interestingly, the depletion
capacitance measured implies that, within the depletion region, the electronic polarizability is strongly enhanced.
